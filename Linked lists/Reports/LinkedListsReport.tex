\documentclass[a4paper,11pt]{article}

\usepackage[utf8]{inputenc}


\begin{document}

\title{
    \textbf{Linked Lists}
}
\author{Dhruba Das}
\date{Spring Fall 2023}

\maketitle

\section*{Introduction}
In this assignment, we are tasked with the creation and benchmarking of a new data structure:  linked lists. There are 3 problems in this assignment: creation of the linked list data structure (i.e Cell structure/its properties), implementation of methods given, and benchmarking against arrays of similar size;

\section*{Cell Structrue}

The cell or node, is an object with a pointer to its next cell, and a value it contains. Below is the code snippet for cell class:

\begin{verbatim}
public class Cell {
        int head;
        Cell tail;
    
        public Cell(int val, Cell tl) {
            head = val;
            tail = tl;
        }
    }
\end{verbatim}
As seen above, a constructor is made for the cell object, which allows us to change its pointer and the value it contains.

\section*{Implementation of the methods}

For each method, I had to keep in mind that I had to traverse the entire list most of the time; this is important since we're only working with pointers, without any actual references to the cells themselves.


\subsection*{Add method}

This was easy to implement as we only had to add an cell with a specific value to the beginning of the list. Below is the code snippet:
 
\begin{verbatim}
public void add(int item){
            Cell newC = new Cell(item, first); 
            first = newC;
        }
\end{verbatim}

\subsection*{Length method}

Here, we set a counter in order to count the number of cells we have traversed (i.e contained in the list) as we traverse the entire list, in order to find the size of the list. Below is the code snippet:
 
\begin{verbatim}
public int length(){
            Cell current = first;
            int counter = 0;
            while(current != null){
                counter++;
                current = current.tail;
            }
            return counter;
        }
\end{verbatim}


\subsection*{Find method}

This method traverses the list in search of a cell which contains the value we are looking for. Below is the code snippet:
 
\begin{verbatim}
public boolean find(int item){
            Cell current = first;
            while(current != null){
                if(current.head == item)
                 return true;
                current = current.tail;
            }
            return false;
        }
\end{verbatim}
The time complexity of this search is O($n$).

\subsection*{Remove method}

This was the most difficult among the methods given to implement; it removes the cell containing the value we are looking for, from the list. In order to do so, we link the previous cell of the current cell we are looking at, to the next cell of the current cell we are looking at. It also contains a special case, for the instance that the value we are looking for is in the first cell. Below is the code snippet:

\begin{verbatim}
 public void remove(int item){
            Cell current = first;
            Cell prev = null;
            while(current != null){
                if(first.head == item){
                    first = first.tail;
                    return;
                }
                if(current.head == item){
                     prev.tail = current.tail;
                     return;
                }
                prev = current;
                current = current.tail;
            }
            return;
        }
\end{verbatim}

\subsection*{Append method}

This method appends one list to another given list. Below is the code snippet:

\begin{verbatim}
 public void append(LinkedList b) {
            Cell current = first;
            while (current.tail != null) {
                current = current.tail;
            }
            current.tail = b.first;
            b.first = null;
        }
\end{verbatim}

\section*{Benchmarks}

\subsection*{Constant list and Increasing list}

I have represented the difference in performances between appending two linked lists, in a table below; note that the size of the list being appended increasing in size:

\begin{table}[h]
\begin{center}
\begin{tabular}{l|c|c}
\textbf{Size of b (n)} & \textbf{Time(ns)}\\
\hline
  200      &  2416 \\
  400      &  2875\\
  800      &  2250\\
  1600      &  2209\\
  3200      &  2125 \\
  6400      &  2417\\
  12800      &  1875\\
  25600      &  2833\\
\end{tabular}
\caption{(For each size, append is carried out for 1000 loops; the min. time (in nanoseconds) is taken; $b$ is the list being appended)}
\label{tab:table1}
\end{center}
\end{table}

\subsection*{Increasing list and Constant list}

Next, below is the table for when the list being appended to is increasing in size:
\begin{table}[h]
\begin{center}
\begin{tabular}{l|c|c}
\textbf{Size of a (n)} & \textbf{Time(ns)}\\
\hline
  200      &  2375\\
  400      &  2259\\
  800      &  3000\\
  1600      &  5584\\
  3200      &  11125\\
  6400      &  21625\\
  12800      &  30833\\
  25600      &  80583\\
\end{tabular}
\caption{(For each size, append is carried out for 1000 loops; the min. time (in nanoseconds) is taken; $a$ is the list being appended to)}
\label{tab:table1}
\end{center}
\end{table}

 And as seen above, in both cases, the time taken depends on the array being appended to, with a time complexity of O($n$). If the list being appended to is fixed, the time taken to traverse the list stays approx. constant, whereas, when the list is growing, the traversal time increases.


\section*{Linked vs Array}
\subsection*{Constant list and Constant array}

I have represented the difference in performances between appending a constant and an increasing linked list, and a constant and an increasing array, in a table below:

\begin{table}[h]
\begin{center}
\begin{tabular}{l|c|c|c}
\textbf{Size(n)} & \textbf{Linked List (ns)} & \textbf{Arrays(ns)}\\
\hline
  200      &  2166 &     2417\\
  400      &  2833 &     3125\\
  800      &  2083 &     542\\
  1600      &  2041 &     833\\
  3200      &  2125 &     1416\\
  6400      &  2250 &     2458\\
  12800      &  2209 &     4750\\
  25600      &  2292 &     9292\\
\end{tabular}
\caption{(For each size, append is carried out for 1000 loops; the min. time (in nanoseconds) is taken)}
\label{tab:table1}
\end{center}
\end{table}

\subsection*{Increasing list and Increasing array}

I have represented the difference in performances between appending an increasing and a consant linked list, and a increasing and a constant array, in a table below:

\begin{table}[h]
\begin{center}
\begin{tabular}{l|c|c|c}
\textbf{Size(n)} & \textbf{Linked List (ns)} & \textbf{Arrays(ns)}\\
\hline
  200      &  2375 &     2458\\
  400      &  2458 &     3125\\
  800      &  2916 &     625\\
  1600      &  4875 &     958\\
  3200      &  8625 &     1541\\
  6400      &  15917 &     2917\\
  12800      &  30333 &     5375\\
  25600      &  93084 &     10583\\
\end{tabular}
\caption{(For each size, append is carried out for 1000 loops; the min. time (in nanoseconds) is taken)}
\label{tab:table1}
\end{center}
\end{table}

As seen from the above tables, the cost of appending arrays is about the same, regardless of whether the array being appended is increasing in size, or the array being appended to is increasing in size.



\subsection*{Allocation}

I have represented the difference in performances between allocating an array and a linked list, in a table below:

\begin{table}[h]
\begin{center}
\begin{tabular}{l|c|c|c}
\textbf{Size} & \textbf{Linked List (ns)} & \textbf{Arrays(ns)}\\
\hline
  200      &  1417 &     167\\
  400      &  1041 &     208\\
  800      &  2792 &     209\\
  1600      &  6458 &     333\\
  3200      &  13125 &     500\\
  6400      &  26791 &     916\\
  12800      &  55333 &     750\\
  25600      &  107208 &     1125\\
\end{tabular}
\caption{(For each size, allocation is carried out for 1000 loops; the min. time (in nanoseconds) is taken)}
\label{tab:table1}
\end{center}
\end{table}
As we can see, the cost of building a linked list is much higher than that of an array.

\section*{Linked List Stack}

Below is the code snippet for the linked list stack implementation:
\begin{verbatim}
public class LLStack{
    Cell first;

	public class Cell {
        int head;
        Cell tail;
    
        public Cell(int val, Cell tl) {
            this.head = val;
            this.tail = tl;
        }
    }  
    public LLStack(int n) {
        first = null;
    }

	public void push(int val) {
		Cell newC = new Cell(val, first);
		first = newC;
	}

	public int pop() throws Exception {
		if (first == null) {
			throw new Exception("Tried to pop empty stack");
		}
		int tmp = first.head;
		first = first.tail;
		return tmp;
	}
}
\end{verbatim}
As we can immediately see, the linked list implementation of the stack data structure is much more efficient and requires a lot less memory allocation, especially when it comes to the push and pop.











\end{document}