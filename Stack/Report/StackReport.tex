\documentclass[a4paper,11pt]{article}

\usepackage[utf8]{inputenc}


\begin{document}

\title{
    \textbf{Stacks: Dynamic VS Static}
}
\author{Dhruba Das}
\date{Spring Fall 2023}

\maketitle

\section*{Introduction: Problems to solve}

There are in total 4 problems I had to solve in this assignment: 
a) Representing an expression as an array of items.
b) Creating the "step" method so that the steps are executed when the "run" method is executed. 
c) Implementing a static stack.
d) Implementing a dynamic stack.

\section*{Representing an expression}

This was quite simple to do, yet I had problems understanding so. At first, I did not understand that we had to return an Item when a method such as Item.Value() was called, as I thought, since the Item was already there, the methods only modified the type and the value of each item. However, since this was not the case, I had to use a constructor which would allow to construct an item ( as stated in the assignment ) and then further implement methods which would allow us to change the value or the type of an item and then return that item.

Code snippet:
\begin{verbatim}
public Item(ItemType t, int v){
        type = t;
        value = v;
    }
 ...
  public static Item Value(int y){
        return new Item(ItemType.VALUE, y);
    }
    public static Item Mul(){
        return new Item(ItemType.MUL, 0);
    }
    public static Item Add(){
        return new Item(ItemType.ADD, 0);
    }
\end{verbatim}

\section*{Executing the steps}
This was actually the easiest part of the assignment, as I only had to develop the step() method such that it executed whatever expression it encountered. It had 5 parts for the five expressions : VALUE, ADD, SUB, MUL and DIV.


\section*{Static stack}
Although I had worked with stacks before, I had never implemented one myself. I had to thoroughly read through the guidelines in the assignment and refer to previous experiences in order to implement the stack. Below is the code snippet for the static stack I implemented:

\begin{verbatim}
public class StaticStack extends Stack{
    int sp = -1;
    int [] array = new int[8];
    public void push(int x){
        if(sp == array.length - 1){
          System.out.println("Stack is full");
         } 
         sp++;
         array[sp] = x; 
    }
    
    public int pop(){
        if(sp == -1){
        System.out.println("Stack is empty");
        }
        int y = array[sp];
        array[sp] = 0;
        sp--;
        return y;
    }
}
\end{verbatim}
As shown above, error messages have been included in cases where the stack might be full or empty.  

\section*{Dynamic Stack}
Quite a bit of the code for the dynamic stack is same as that of the static stack, as both are initialised with fixed-size arrays. It is in the pop() and push() methods where changes have been made. Below are the changes which were made:

\begin{verbatim}
public void push(int x){
        if(sp == array.length - 1){
          int [] newArray = new int[array.length * 2];
          for(int i = 0; i < newArray.length; i++){
            newArray[i] = array[i];
          }
          array = newArray;
...
public int pop(){
        int counter = 0;
        if(sp + 1 < array.length/2)
          counter--;
        else
          counter = 10;
        
        if(counter == 0){
         int [] newArray = new int [array.length/2];
         for(int i = 0; i < newArray.length; i++){
           newArray[i] = array[i];
        }
        array = newArray;
   } ...
\end{verbatim}
As seen above, I have doubled the size of the array in case the stack is full, and I have set up a counter to check whether the extra space is used in stack after popping; if not, the size of the stack is halved.


\section*{Benchmarks}

I had a little bit of a tough time doing this, as I have never benchmarked a method or a code's efficiency before. I have represented the data I have collected below in the form of a table:

\begin{table}[h]
\begin{center}
\begin{tabular}{l|c|c}
\textbf{Stack} & \textbf{Initial size} & \textbf{Min. time taken (ms)}\\
\hline
  Dynamic     &  8 &     24\\
  Static      &  8 &     20\\
\end{tabular}
\caption{Dynamic and Static; same initial size; 1000 runs each;  runtime in milliseconds;}
\label{tab:table}
\end{center}
\end{table}

As seen in the table above, the static stack is faster than the dynamic, but not by much, with only a difference of 4ms.
Below listed is the code which I used to benchmark (had to run it a 1000 times again and pick the minimum times in a different file).

\begin{verbatim}
        for(int i = 0; i < time.length; i++){
        Calculator calc = new Calculator(expr);
        res = calc.run();
        long endTime = System.currentTimeMillis();
        time[i] = endTime - startTime;
        }

        long minTime = 0;

        for(int i = 0; i < time.length - 1; i++){
            minTime = time[i];
            if(minTime > time[i + 1])
             minTime = time[i + 1];
        }
         

        System.out.println(" Calculator: res = " + res);
        System.out.println(" Minimum time taken (ms): " + minTime);
  }
  
}
\end{verbatim}





\end{document}
