\documentclass[a4paper,11pt]{article}

\usepackage[utf8]{inputenc}


\begin{document}

\title{
    \textbf{Searching algorithms : Linear VS Binary VS Final}
}
\author{Dhruba Das}
\date{Spring Fall 2023}

\maketitle

\section*{Introduction: Problems to solve}

There are in total 3 problems I had to solve in this assignment: 
a) Implementing the Binary Search algorithm.
b) Testing the Binary Search algorithm.
c) Creating and testing all 3 algorithms including the Final version.

\section*{Binary Search}

This was actually the easiest part of the assignment. I had learned about this algorithm in a previous course and could easily refresh my memory by looking back. Of course, I had to also read through the guidelines to see if any modifications were to be made. Below is the code snippet:

\begin{verbatim}
public static Boolean search(int[] array, int key)
    {
        int first = 0;
        int last = array.length-1;
        while (true)
        {
            int index = (last+first)/2;

            if (key == array[index])
                return true;
            if (key > array[index])
                first = index+1;
            else if (key < array[index])
                last = index-1;
            if (first > last)
                return false;
        }
	
    }
\end{verbatim}

\section*{Testing the binary search algorithm}
This bit was also quite easy as I could easily test it using the benchmark program which was provided. Here, we only benchmark the binary search against the linear search algorithm. Below are the values I obtained from the benchmarking program:

\begin{table}[h]
\begin{center}
\begin{tabular}{l|c|c}
\textbf{Size of array (n)} & \textbf{Linear (ns)} & \textbf{ Binary (ns)}\\
\hline
  100     &  16 &     23\\
  200     &  32 &     27\\
  300     &  46 &     30\\
  400     &  63 &     32\\
  500     &  75 &     34\\
  600     &  92 &     35\\
  700     &  106 &     37\\
  800    &  119 &     37\\
  900    &  138 &     38\\
  1000     &  153 &     38\\
  1100     &  164 &     39\\
  1200     &  181 &     40\\
  1300     &  196 &     41\\
  1400     &  212 &     42\\
  1500     &  230 &     42\\
  1600     &  240 &     42\\
\end{tabular}
\caption{Binary and linear; same size for each ; 1000 runs each;  runtime for 1 run in nanoseconds;}
\label{tab:table}
\end{center}
\end{table}

I tested the binary search algorithm for an array with a size of 1 million, and I got a runtime of 106 ns.



\section*{Creating and testing the final version}

\subsection*{The final version}
I hadn't seen this algorithm before, but some of my peers have worked with this algorithm before. It's known as the "smart search" algorithm. It was not difficult to implement, as I only had to keep track of items in two arrays, compare them, move on and at the end, return a counter which gave us the number of duplicates between the two arrays. Below is the code snippet for the algorithm:


\begin{verbatim}
public static int search(int[] array1, int[] array2) 
    {
        int counter = 0;
        int i = 0;
        int j = 0;

        while (i<array1.length-1 && j<array2.length-1)
        {
            if (array1[i] > array2[j])
                j++;
            if (array1[i] < array2[j])
                i++;
            else
            {
                counter++; 
                i++; 
                j++;
            }
        }
        return counter;
    } 
\end{verbatim} 

\subsection*{Testing all three algorithms }
Here, I only had to make minor changes to the given benchmark program ( to add the in the final search algorithm ). Below is the snippet for the code I added in:

\begin{verbatim}
 for (int i = 0; i < k; i++) {
		long t0 = System.nanoTime();
		FinalSearch.search(array1, array2);
		long t1 = System.nanoTime();
		double t = (t1 - t0);
		if (t < min)
		    min = t;
	    }

	    System.out.printf("%8.0f\n" , (min/loop));
	}
    }
\end{verbatim}

Below I have listed the values I obtained from the benchmarking program ( NOTE: All values shown as 0 are values of less than 1 ns ).

\begin{table}[h]
\begin{center}
\begin{tabular}{l|c|c|c}
\textbf{Size of array (n)} & \textbf{Linear (ns)} & \textbf{Binary (ns)} & \textbf{Final (ns)}\\
\hline
  100     &  0 &     0 &     0\\
  200     &  1 &     0 &     0\\
  300     &  1 &     0 &     0\\
  400     &  2 &     0 &     0\\
  500     &  4 &     0 &     0\\
  600     &  5 &     1 &     0\\
  700     &  8 &     1 &     0\\
  800    &  10 &     1 &     0\\
  900    &  13 &     1 &     0\\
  1000     &  15 &     1 &    0\\
  1100     &  18 &     1 &    0\\
  1200     &  22 &     1 &    0\\
  1300     &  26 &     2 &    0\\
  1400     &  30 &     2 &    0\\
  1500     &  35&      3 &     0\\
  1600     &  39 &     3 &     0\\
\end{tabular}
\caption{Binary, linear and final; same size for each ; 1000 runs each;  runtime for 1 run in nanoseconds;}
\label{tab:table}
\end{center}
\end{table}

Above is the time taken for all three of the algorithms to search for duplicates in two given sorted arrays.


\section*{Conclusion}

As it can be seen from the above data, Binary searching is far more efficient than Linear searching as we know approximately where to look, and smart search is even more so than Binary (when it comes to looking for duplicates or searching through an array in general). As of right now, sorting seems more cost efficient to me when it comes to searching through arrays/data.





\end{document}
